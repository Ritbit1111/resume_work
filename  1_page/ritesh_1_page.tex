\documentclass[a4paper, 10pt]{article}
%-----------------------------------------------------------
\usepackage[top=0.5in, bottom=0.5in, left=0.55in, right=0.61in]{geometry}

\usepackage{graphicx}
\usepackage{url}
\usepackage{palatino}
\usepackage{tabularx}
\fontfamily{SansSerif}
\hyphenpenalty=10000
\selectfont
\usepackage[utf8]{inputenc}
\usepackage{amsmath}
\usepackage[T1]{fontenc}
\linespread{0.9}
\usepackage
[utf8]
{inputenc}

\usepackage{color}
\definecolor{mygrey}{gray}{0.7}
%\definecolor{mygrey}{HTML}{D3D3D3}
\setlength{\tabcolsep}{0in}
\newcommand{\isep}{-2 pt}
\newcommand{\lsep}{-0.5cm}
\newcommand{\psep}{-0.6cm}
\renewcommand{\labelitemii}{$\circ$}
\pagestyle{empty}
\newcommand{\resitem}[1]{\item #1 \vspace{-2pt}}
\newcommand{\resheading}[1]{{\normalsize \colorbox{mygrey}{\begin{minipage}{0.995\textwidth}{\begin{center}				\textbf{#1 \vphantom{p\^{E}}}\end{center}}\end{minipage}}}}
\newcommand{\ressubheading}[3]{
\begin{tabular*}{6.62in}{l @{\extracolsep{\fill}} r}
	\textsc{{\textbf{#1}}} & \textsc{\textit{[#2]}} \\
\end{tabular*}\vspace{-8pt}}
\begin{document}
\vspace*{135pt}
\setlength{\leftmargini}{1em}
\vspace{-1.32cm}
\noindent \resheading{\textbf{PROFFESSIONAL EXPERIENCE} }\\[\lsep]
\vspace{5pt}
\begin{itemize} \itemsep \isep 
\vspace{-0.03cm}
\item \textbf{\large Edelweiss Securities Limited | Trading Technology Team}\hfill\emph{(May'18-July'18)} \\
% \textit{Guide: Prof. Narendra Shiradkar, EE Dept. IIT Bombay}\hfill{\em{(Jul'19-Present)}}\\
\vspace{2pt}
	\begin{itemize}\itemsep \isep
     \vspace{-0.70cm}
    \item Designed \textbf{UX} and \textbf{UI} for easy access to \textbf{Transaction Cost Analysis (TCA)} report to help traders get actionable \textbf{insights} to enhance \textbf{trading} related \textbf{execution quality}, compliance and management reporting capabilities
\item Implemented \textbf{login authorization} and \textbf{dynamic forms} to query single day and multiple day TCA report based on date, account ID, portfolio and instrument with \textbf{download link} to summary file on \textbf{Django framework}
\item Constructed infrastructure for \textbf{logging errors}, warnings and regular django server info for future \textbf{debugging}
	\end{itemize}
\end{itemize}
\vspace{-0.2cm}
\noindent \resheading{\textbf{RESEARCH EXPERIENCE AND SURVEY} }\\[\lsep]
\vspace{5pt}
\begin{itemize} \itemsep \isep 
\vspace{-0.05cm}

\item \textbf{\large Data Driven Techniques to Predict Performance Loss of PV Plants | Master's Thesis}\\
%\item \textbf{\large Data Driven Techniques to predict PV Plants Performance Loss} | Masters' Thesis \hfill \emph{(Jul'19-Present)}\\
 \textit{Guide: Prof. Narendra Shiradkar, EE Dept. IIT Bombay}\hfill{\em{(Jul'19-Present)}}\\
%{\flushleft \em{Guide: Prof. Narendra Shiradkar, EE Dept. IIT Bombay}\hfill{\em{[June '19]}}}
%   \emph{Guide: Prof. Girish Kumar, IIT Bombay}
% 	\hfill \emph{Dual Degree Project}\\[-0.6cm]
\vspace{2pt}
	\begin{itemize}\itemsep \isep
	\vspace{-0.7cm}
		\item Developing \textbf{data driven techniques} for predicting the degradation rates \& future revenues of solar PV plants
%	\item Designing predictive analytics tools capable of handling Big Data for extracting the performance loss rate of PV plants from time series data of I-V measurements.
%	\item  Designing predictive analytics tools for handling Big Data in performance loss rate extraction from time series I-V data% 
%	of I-V measurements
\item Building \textbf{predictive analytics} tools capable of handling \textbf{big data} for extracting the \textbf{performance degradation} rate (with confidence bounds) of solar PV plants from time series data of current-voltage(I-V) measurements.
%		\item Aimimg to use data science to perform predicitve analysis of reliability of solar modules
%	\item Creating models for solar cells to determine the dependency f degradation rate on various parameters
%	\item Implemented five parameters single diode model for solar cell by extracting the parameters from module datasheet and implicitly solving ideal diode equation using python
%		\item Implemented five parameters single diode model for solar cell by using module datasheet values on python
		\item  Implemented a \textbf{five parameter single diode model} for PV modules in Python that can predict the module power at \textbf{any irradiance and temperature} by extracting the parameters from the module datasheet
	\item Utilized Bokeh server to plot the I-V curve (with interactive sliders) by numerically solving the diode equation
	%	\item Keywords to add : predictive analytics, data science, pv performance, modelling and simulation, reliability,
    \end{itemize}

\vspace{-0.1cm}
	
\item \textbf{\large PV Module Field Survey in Leh | NCPRE, IIT Bombay} \hfill \emph{(Jan'19-May'19)} \\
\vspace{2pt}
%   \emph{Guide: Prof. Rajbabu Velmurugan}
% 	\hfill \emph{Supervised Research Expedition}\\[-0.6cm]
	\begin{itemize}\itemsep \isep
    \vspace{-0.70cm}
\item Inspected solar plant installations and carried out survey on their performance degradation in Leh
\item Surveyed \textbf{88 modules at 3 sites in Ladakh} region and carried out module and string level \textbf{I-V characterization}, IR thermography for \textbf{hotspot} detection and visual imaging to capture \textbf{cracks} and physical damages of the cells
\item Calculated average performance degradation rate per year to be \textbf{1.42\%, 3.32\% and 3.97\%}	
\end{itemize} 
\end{itemize}
\vspace{-0.2cm}
\noindent \resheading{\textbf{MAJOR PROJECTS } }\\[\lsep]
\vspace{5pt}
% \vspace{m}
\begin{itemize} \itemsep \isep 
\vspace{-0.01cm}
\item \textbf{\large Solar Module Mounting Orientation} | \textit{Course: Design and Evaluation of PV Power Plants} \hfill \emph{(Mar'19-May'19)} \\
    \vspace{2pt}
	\begin{itemize}\itemsep \isep
	\vspace{-0.70cm}
 \item Determined the best possible orientation of solar panel for \textbf{maximum power output} in different regions.
\item Performed \textbf{parametric analysis on System Advisor Model software} by varying tilt and azimuthal angle% in northern, southern and equatorial regions for summers and winters.
%  \item Concluded that optimized tilt angle was latitude angle and panel should face south in northern hemisphere and vice versa
\item Concluded that optimal tilt angle is latitude angle and optimal azimuth is 180 in north and 0 in south
%\item Observed for equatorial regions optimal azimuth angle changed seasonally and axis tracking was effective
 	\end{itemize}
\vspace{-0.18cm}

\item \textbf{\large Portable Solar cum Vibration Energy Harvesting Phone Charger} |  \textit{Design Lab} \hfill \emph{(Jan'18-Apr'18)} \\
%   \emph{Guide: Prof. Joseph John}
% 	\hfill \emph{Electronic Design Lab Project}\\[-0.6cm]
    \vspace{2pt}
\begin{itemize}\itemsep \isep
	\vspace{-0.70cm}
			\item Prototyped and tested working model of solar cum vibration charger with optimized size and efficiency
\item Designed a suitable \textbf{AC-DC} converter and a \textbf{DC-DC Boost} converter for vibration and solar circuit output
\end{itemize}
\vspace{-0.18cm}

\item \textbf{\large Power Amplifier Design} |  \textit{Course: Solid State Microwave Devices} \hfill \emph{(Mar'19-May'19)} \\
    \vspace{2pt}
	\begin{itemize}\itemsep \isep
	\vspace{-0.70cm}
		\item Simulated a 2 stage power amplifier with \textbf{matching \& bias-T} circuits with unilateral design approach in ADS 
	\item Designed, fabricated \& tested the PCB using Vector Network Analyzer for \textbf{gain and bandwidth} specifications
	\end{itemize}

\vspace{-0.18cm}

\item \textbf{\large Maze Solver} | \textit{ Summer School of Code, WnCC IIT Bombay} \hfill \emph{(May'16-Jul'16)} \\
    \vspace{2pt}
	\begin{itemize}\itemsep \isep
	
	\vspace{-0.70cm}
		\item Implemented command line \textbf{Image Processing Project} on Python platform assisted by \textbf{OpenCV} library
\item Used thresholding, filters, \textbf{contour extraction}, and \textbf{thinning} (one pixel width) to get a path from start to end
	\end{itemize}

% \vspace{-0.15cm}

\end{itemize}

\vspace{-0.2cm}
\noindent \resheading{\textbf{POSITIONS OF RESPONSIBILITY}}\\[\lsep]
\vspace{5pt}
\begin{itemize}

\itemsep \isep 
\item \textbf{\large Teaching Assistant | Course: Reliability and Failure Analysis} \hfill \emph{(Jul'19-Present)} \\[-0.6cm]
%\vspace{-0.1cm}
%\vspace{-0.02cm}
	\begin{itemize}\itemsep \isep
		\item Developing an online portal using interactive Python library \textbf{Bokeh} \& Jupyter notebooks that would provide the students \textbf{personalized random failure data} of various distributions for their course project \textbf{(Virtual Lab)}
%			\item Developed online portal for 40+ students to access personalized random failure data for their course project% to predict the nature of failure
		\item Generated \textbf{artificial} random data of \textbf{Normal, Weibull and Lognormal} distributions for modeling \& simulation

\end{itemize}
\vspace{-0.22cm}
 \item \textbf{\large Campaigning Coordinator| {Abhuyday, Social Body IIT Bombay}} \hfill\emph{(2016)}
 %\newline\emph{NSS is IITB's largest student volunteer body, serving 100k+ nationwide via public welfare activities}
 \vspace{-0.3cm}
 
 \begin{itemize}\itemsep \isep
 	\item Led volunteer weekends at schools for the underprivileged to instil \textbf{computer basics and career counselling}	
 	\item Co-ordinated and volunteered ANTARCHAKSHU, St.\ Xavier’s XRCVC’s initiative with a motive to demand from the government and people \textbf{equal accessibility} to science education for \textbf{visually challenged people}
 \end{itemize}
\end{itemize}

\vspace{-0.20cm}

\noindent \resheading{\textbf{TECHNICAL SKILLS} }
%%%%%%%%%%%%%%%%%%%%%%%%%%%%%%%%%%%%%%%%%%%%%%%%%%%%%%%%%%%%%%%%%%%%%%%%%%%%%%%%%%%%%%%%%%%%%%%%55
%\begin{itemize}
%	\setlength\itemsep{0.1pt}
%	\setlength\parskip{0.1pt}
%	%\item Secured \textbf{third position} in Hostel Chess General Championship \hfill \emph{('17-'18)} 
%	\item  Programming Languages : \textbf{Python, C++}, VHDL
%	\item Tools : MATLAB, SAM, Cadence Virtuoso, Quartus, ADS, Bokeh, Django, OpenCV
%\end{itemize}
%%%%%%%%%%%%%%%%%%%%%%%%%%%%%%%%%%%%%%%%%%%%%%%%%%%%%%%%%%%%%%%%%%%%%%%%%%%%%%%%%%%%%%%%%%%%%55%%%5
\begin{tabular}{m{4.5cm}   m{14cm}} 
	\vspace{0.1cm}
	\textbf{Programming Languages} & \vspace{0.1cm} Python, C++, VHDL, Assembly\\ 
	\textbf{Tools \& Libraries} & MATLAB, R, Cadence Virtuoso, Quartus, ADS, Bokeh, Django, PVLib, Pandas, Numpy \\  
\end{tabular}\\
%\vspace{-0.2cm}
\noindent \resheading{\textbf{EXTRA CURRICULAR ACTIVITIES} }
\vspace{-0.4cm}
\begin{itemize}
\setlength\itemsep{0.1pt}
\setlength\parskip{0.1pt}
%\item Secured \textbf{third position} in Hostel Chess General Championship \hfill \emph{('17-'18)} 
\item Bestowed with a \textbf{Black belt (1st Dan)} at an age of 12 in Shotokan Karate after regular training of \textbf{4 years} \hfill\emph{(2009)}
%\item Member of Gold medal recieving squad in Badminton General Championship among 12 hostels \hfill\emph{(2018)}
\item Recieved \textbf{Gold medal in Badminton} inter hostel General Championship (11 participants) in IIT Bombay\hfill\emph{(2018)}
\item Awarded \textbf{Silver medal for Street Play} Mann ki Bhadas" in Freshmen cultural competitions, IIT Bombay \hfill\emph{(2015)}
\item Pursuing 50 hours official \textbf{German language course} provided by International Relation Cell, IIT Bombay \hfill\emph{(2019)}
\end{itemize}
%\vspace{-0.15cm}
%\vspace{-0cm}
%
%\noindent \textbf{Hobbies} \\
%Badminton |Trekking | Travelling | Cooking 
\end{document}

