%% start of file `template.tex'.
%% Copyright 2006-2013 Xavier Danaux (xdanaux@gmail.com).
%
% This work may be distributed and/or modified under the
% conditions of the LaTeX Project Public License version 1.3c,
% available at http://www.latex-project.org/lppl/.


\documentclass[11pt,a4paper,roman]{moderncv}        % possible options include font size ('10pt', '11pt' and '12pt'), paper size ('a4paper', 'letterpaper', 'a5paper', 'legalpaper', 'executivepaper' and 'landscape') and font family ('sans' and 'roman')

% modern themes
\moderncvstyle{banking}                            % style options are 'casual' (default), 'classic', 'oldstyle' and 'banking'
\moderncvcolor{blue}                                % color options 'blue' (default), 'orange', 'green', 'red', 'purple', 'grey' and 'black'
%\renewcommand{\familydefault}{\sfdefault}         % to set the default font; use '\sfdefault' for the default sans serif font, '\rmdefault' for the default roman one, or any tex font name
\nopagenumbers{}                                  % uncomment to suppress automatic page numbering for CVs longer than one page

% character encoding
\usepackage[utf8]{inputenc}
\usepackage{fontawesome}
\usepackage{tabularx}
\usepackage{ragged2e}
% if you are not using xelatex ou lualatex, replace by the encoding you are using
%\usepackage{CJKutf8}                              % if you need to use CJK to typeset your resume in Chinese, Japanese or Korean

% adjust the page margins
\usepackage[scale=0.8]{geometry}
\usepackage{multicol}
%\setlength{\hintscolumnwidth}{3cm}                % if you want to change the width of the column with the dates
%\setlength{\makecvtitlenamewidth}{10cm}           % for the 'classic' style, if you want to force the width allocated to your name and avoid line breaks. be careful though, the length is normally calculated to avoid any overlap with your personal info; use this at your own typographical risks...

\usepackage{import}

% personal data
\name{Ritesh}{Kumar}
% \title{Curriculum Vitae}                               % optional, remove / comment the line if not wanted
\address{4086, Hostel 18, IIT Powai}{}{}% optional, remove / comment the line if not wanted; the "postcode city" and and "country" arguments can be omitted or provided empty
% \phone[mobile]{909-839-3097}                   % optional, remove / comment the line if not wanted
% \phone[fixed]{01234 123456}                    % optional, remove / comment the line if not wanted
%\phone[fax]{+3~(456)~789~012}                      % optional, remove / comment the line if not wanted
% \email{xpan1@swarthmore.edu}                               % optional, remove / comment the line if not wanted
% \homepage{shawnpan.me}                         % optional, remove / comment the line if not wanted
% \extrainfo{}                 % optional, remove / comment the line if not wanted
%\photo[64pt][0.4pt]{picture}                       % optional, remove / comment the line if not wanted; '64pt' is the height the picture must be resized to, 0.4pt is the thickness of the frame around it (put it to 0pt for no frame) and 'picture' is the name of the picture file
%\quote{Some quote}                                 % optional, remove / comment the line if not wanted

% to show numerical labels in the bibliography (default is to show no labels); only useful if you make citations in your resume
%\makeatletter
%\renewcommand*{\bibliographyitemlabel}{\@biblabel{\arabic{enumiv}}}
%\makeatother
%\renewcommand*{\bibliographyitemlabel}{[\arabic{enumiv}]}% CONSIDER REPLACING THE ABOVE BY THIS

% bibliography with mutiple entries
%\usepackage{multibib}
%\newcites{book,misc}{{Books},{Others}}
  
\newcommand*{\customcventry}[7][.25em]{
  \begin{tabular}{@{}l} 
    {\bfseries #4}
  \end{tabular}
  \hfill% move it to the right
  \begin{tabular}{l@{}}
     {\bfseries #5}
  \end{tabular} \\
  \begin{tabular}{@{}l} 
    {\itshape #3}
  \end{tabular}
  \hfill% move it to the right
  \begin{tabular}{l@{}}
     {\itshape #2}
  \end{tabular}
  \ifx&#7&%
  \else{\\%
    \begin{minipage}{\maincolumnwidth}%
      \small#7%
    \end{minipage}}\fi%
  \par\addvspace{#1}}

\newcommand*{\customcvproject}[4][.25em]{
%   \vfill\noindent
  \begin{tabular}{@{}l} 
    {\bfseries #2}
  \end{tabular}
  \hfill% move it to the right
  \begin{tabular}{l@{}}
     {\itshape #3}
  \end{tabular}
  \ifx&#4&%
  \else{\\%
    \begin{minipage}{\maincolumnwidth}%
      \small#4%
    \end{minipage}}\fi%
  \par\addvspace{#1}}

\setlength{\tabcolsep}{12pt}

%----------------------------------------------------------------------------------
%            content
%----------------------------------------------------------------------------------
\begin{document}
%\begin{CJK*}{UTF8}{gbsn}                          % to typeset your resume in Chinese using CJK
%-----       resume       ---------------------------------------------------------
%\makecvtitle
\vspace*{4.2cm}

%\begin{center}
%\begin{tabular}{ c c c c }
% \faGlobe\enspace mysite.me & \faEnvelopeO\enspace pchatter@andrew.cmu.edu & \faGithub\enspace yourgithub &  \faMobile\enspace 928-409-2740\\  
%\end{tabular}
%\end{center}

\section{INTERNSHIP AND SURVEY}
{\customcvproject{Edelweiss Financial Services Limited}{May 2018 - July 2018}
  {\begin{itemize}
		\item TCA report on Django server based UI
		\item Added more to TCA report for prediction.
		\item Helped traders to  see and download their TCA report file of their transactions for their clients
	\end{itemize}
  }
}
{\customcvproject{Leh Solar Power Plant Survey}{June 2019 }
	{\begin{itemize}
			\item Examine the Leh Ladakh conditions for solar growth for further scope in solar setup in Leh
			\item Survey of 4 solar plants of 12kWp units
			\item Percentage degradation data
		\end{itemize}
	}
}

\section{PROJECTS}

{\customcvproject{Degradation rate of solar power plants | Dual Degree Project}{July 2019 - Present}
  {To create models for solar cells and find the dependency of parameters on the degradation rate}
  {Work done till date:}
  {\begin{itemize}
    \item Implemented single diode model for solar cell with five parameters 
    \item Used Bokeh server to plot the I-V curve interactively with varying parameters sliders.
  \end{itemize}
  }
}

{\customcvproject{Solar module mounting orientation and axis tracking effect}{March 2019– April 2019}
{\begin{itemize}
  \item Determined the best possible orientation of solar panel for maximum power output in different regions.
  \item Performed parametric analysis on System Advisor Model software by varying tilt and azimuthal angle.
  \item Conclusions after analysis in 3 different regions (northern, southern and equatorial) each in 2 seasons (summers and winters):
  \item Tilt angle should be equal to the latitude angle
  \item Solar panel should be facing south in northern hemisphere and vice versa but for equatorial region the azimuthal angle for maximum output changes with summer and winter season
\end{itemize}
}

{\customcvproject{Image compression using wavelet transform algorithm}{March 2019 – April 2019}
{\begin{itemize}
\item Implemented image compression algorithm using 4-taps, 2-D Daubechies Wavelet Transform on 512 x 512 grayscale image and reconstructed the image using Inverse Daubechies Wavelet Transform
\item Implemented whole system on Cyclone IV-E Altera FPGA using Nios II processor in platform designer interfaced with SDRAM module on-board which is capable of storing input and output image data of large sizes
\item Applied low pass and high pass filtering followed by downsampling by 2 on rows and columns sequentially to obtain LL, LH, HL and HH image components
\item Implemented thresholding on image and performed Huffman encoding to obtain compressed image data which is decoded and then reconstructed back
\end{itemize}
}
}

{\customcvproject{Power Amplifier design}{March 2019 – April 2019}
	{\begin{itemize}
			\item Simulated in ADS a 2 stage power amplifier with matching \& bias T circuits with unilateral design approach
			\item Designed, fabricated \& testsd the PCB using Vector Network Analyzer for gain and bandwidth specifications
		\end{itemize}
		}
}

{\customcvproject{Modelling gesture control}{March 2019 – April 2019}
	{\begin{itemize}
			\item Modelled 3-D Gesture Control using ADXL345 Digital Accelerometer interfaced with Arduino board
			\item Estimated inclination angle of the three axes with an error of less than 5\% and plotted the same in real time
		\end{itemize}
	}
}

{\customcvproject{IITB-RISC Microprocessor design}{March 2019 – April 2019}
	{\begin{itemize}
			\item Designed a 16-bit system with 8 registers having multi-cycle point to point communication infrastructure
			\item Synthesized VHDL code integrating the controller-FSM and data path for FPGA demonstration
		\end{itemize}
	}
}

{\customcvproject{Portable Solar cum Vibration Energy Harvesting Mobile Charger}{March 2019 – April 2019}
	{\begin{itemize}
			\item Designed a suitable AC-DC converter and a DC-DC Boost converter for vibration and solar circuit output
			\item Prototyped and tested working model of the charger with optimized size and performance
		\end{itemize}
	}
}

\section{AREAS OF INTEREST}
\begin{itemize}
	\item Solar System Design, Reliability of devices, Data Structure and Algorithm
\end{itemize}
\section{Extra curricular Activities}
\begin{minipage}{\maincolumnwidth}%
	\small{
    	\begin{itemize}
          \item Gold, GC Badminton
          \item Flute class
          \item German Class
          \item Black belt 1st Dan Shotokan Karate
		\end{itemize}}%
\end{minipage}%
}
.
% Publications from a BibTeX file without multibib
%  for numerical labels: \renewcommand{\bibliographyitemlabel}{\@biblabel{\arabic{enumiv}}}% CONSIDER MERGING WITH PREAMBLE PART
%  to redefine the heading string ("Publications"): \renewcommand{\refname}{Articles}
\nocite{*}
\bibliographystyle{plain}


% Publications from a BibTeX file using the multibib package
%\section{Publications}
%\nocitebook{book1,book2}
%\bibliographystylebook{plain}
%\bibliographybook{publications}                   % 'publications' is the name of a BibTeX file
%\nocitemisc{misc1,misc2,misc3}
%\bibliographystylemisc{plain}
%\bibliographymisc{publications}                   % 'publications' is the name of a BibTeX file

%-----       letter       ---------------------------------------------------------

\end{document}


%% end of file `template.tex'.
